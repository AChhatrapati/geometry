
%\renewcommand{\thefigure}{\theenumi.\arabic{figure}}
\begin{figure}[!ht]
\centering
\resizebox{\columnwidth}{!}{%\documentclass{article}
%\usepackage[utf8]{inputenc}
%\usepackage{tikz}
%\usepackage{tkz-euclide}
%\begin{document}

\begin{tikzpicture}
[scale=1,>=stealth,point/.style={draw,circle,fill = black,inner sep=0.5pt},]

%Triangle ABC
\def\a{6}
\def\b{4}
\def\c{5}
 
%Coordinates of A
%\def\p{(\a^2 + \c^2 - \b^2)/ (2*\a)}
\def\p{45/12}
\def\q{{sqrt(\c^2-\p^2)}}

%Labeling points
\node (A) at (\p,\q)[point,label=above right:$A$] {};
\node (B) at (0, 0)[point,label=below left:$B$] {};
\node (C) at (\a, 0)[point,label=below right:$C$] {};

%Foot median AM

\node (M) at (\a/2,0)[point,label=above right:$M$] {};

%Drawing triangle ABC
\draw (A) -- node[left] {$\textrm{c}$} (B) -- node[below] {$\textrm{a}$} (C) -- node[above,xshift=2mm] {$\textrm{b}$} (A);

%Drawing median AM
\draw (A) -- node[left] {$\textrm{}$}(M);

%Drawing and marking angles
\tkzMarkAngle[fill=orange!40,size=0.5cm,mark=](C,B,A)
\tkzLabelAngle[pos=-0.9](A,B,C){$\alpha$}

\end{tikzpicture}

\begin{tikzpicture}
[scale=1,>=stealth,point/.style={draw,circle,fill = black,inner sep=0.5pt},]

%Triangle PQR
\def\p{6}
\def\q{4}
\def\r{5}
 
%Coordinates of P
%\def\x{(\p^2 + \r^2 - \q^2)/ (2*\p)}
\def\x{45/12}
\def\z{{sqrt(\r^2-\x^2)}}

%Labeling points
\node (P) at (\x,\z)[point,label=above right:$P$] {};
\node (Q) at (0, 0)[point,label=below left:$Q$] {};
\node (R) at (\p, 0)[point,label=below right:$R$] {};

%Foot of median

\node (N) at (\p/2,0)[point,label=above right:$N$] {};

%Drawing triangle PQR
\draw (P) -- node[left] {$\textrm{r}$} (Q) -- node[below] {$\textrm{p}$} (R) -- node[above,xshift=2mm] {$\textrm{q}$} (P);

%Drawing median PN
\draw (P) -- node[left] {$\textrm{}$}(N);

%Drawing and marking angles
\tkzMarkAngle[fill=orange!40,size=0.5cm,mark=](R,Q,P)
\tkzLabelAngle[pos=-0.9](P,Q,R){$\theta$}

\end{tikzpicture}
%\end{document}
}
\caption{$\triangle ABC$ and $\triangle PQR$ by Latex-Tikz}
\label{fig:8.1.36_triangle_latex}	
\end{figure}
%
%
%\renewcommand{\thefigure}{\theenumi}
%

\item {\em Construction: }The coordinates of the various points of triangle ABC in Fig. \ref{fig:8.1.36_triangle_latex} are
%\label{}
%
\begin{align}
\vec{B} &= \myvec{0\\0} ,
\label{eq:8.1.36_constr_b}
\\
 \vec{C} &= \myvec{a\\0}, 
\label{eq:8.1.36_constr_c}
\end{align}

$\because \vec{M}$ is the midpoint of $BC$,
\begin{align}
\vec{M}= \frac{\vec{B}+\vec{C}}{2} =\myvec{a/2\\0},
\label{eq:8.1.36_constr_m}
\end{align}
%
$\triangle PQR$ is a horizontal translation of $\triangle ABC$.  Hence, if 
\begin{align}
\vec{Q}= \myvec{q\\0},
\label{eq:8.1.36_constr_q}
\end{align}
\begin{align}
\vec{P}= \vec{A} + \vec{Q}
\\
\vec{R}= \vec{C} + \vec{Q}
\end{align}

%



