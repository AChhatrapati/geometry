\iffalse
\let\negmedspace\undefined
\let\negthickspace\undefined
\documentclass[journal,12pt,twocolumn]{IEEEtran}
\usepackage{cite}
\usepackage{amsmath,amssymb,amsfonts,amsthm}
\usepackage{algorithmic}
\usepackage{graphicx}
\usepackage{textcomp}
\usepackage{xcolor}
\usepackage{txfonts}
\usepackage{listings}
\usepackage{enumitem}
\usepackage{mathtools}
\usepackage{gensymb}
\usepackage[breaklinks=true]{hyperref}
\usepackage{tkz-euclide} % loads  TikZ and tkz-base
\usepackage{listings}
%\usepackage{gvv}

\newtheorem{theorem}{Theorem}[section]
\newtheorem{problem}{Problem}
\newtheorem{proposition}{Proposition}[section]
\newtheorem{lemma}{Lemma}[section]
\newtheorem{corollary}[theorem]{Corollary}
\newtheorem{example}{Example}[section]
\newtheorem{definition}[problem]{Definition}
\newcommand{\BEQA}{\begin{eqnarray}}
\newcommand{\EEQA}{\end{eqnarray}}
\newcommand{\define}{\stackrel{\triangle}{=}}
\theoremstyle{remark}
\newtheorem{rem}{Remark}

\begin{document}
\bibliographystyle{IEEEtran}
\vspace{3cm}
\title{
%	\logo{
Assignment 1
%	}
}
\author{ Barath Surya M (EE22BTECH11014)}
\maketitle
\newpage
\bigskip
\renewcommand{\thefigure}{\theenumi}
\renewcommand{\thetable}{\theenumi}

\providecommand{\pr}[1]{\ensuremath{\Pr\left(#1\right)}}
\providecommand{\prt}[2]{\ensuremath{p_{#1}^{\left(#2\right)} }}        % own macro for this question
\providecommand{\qfunc}[1]{\ensuremath{Q\left(#1\right)}}
\providecommand{\sbrak}[1]{\ensuremath{{}\left[#1\right]}}
\providecommand{\lsbrak}[1]{\ensuremath{{}\left[#1\right.}}
\providecommand{\rsbrak}[1]{\ensuremath{{}\left.#1\right]}}
\providecommand{\brak}[1]{\ensuremath{\left(#1\right)}}
\providecommand{\lbrak}[1]{\ensuremath{\left(#1\right.}}
\providecommand{\rbrak}[1]{\ensuremath{\left.#1\right)}}
\providecommand{\cbrak}[1]{\ensuremath{\left\{#1\right\}}}
\providecommand{\lcbrak}[1]{\ensuremath{\left\{#1\right.}}
\providecommand{\rcbrak}[1]{\ensuremath{\left.#1\right\}}}
\newcommand{\sgn}{\mathop{\mathrm{sgn}}}
\providecommand{\abs}[1]{\left\vert#1\right\vert}
\providecommand{\res}[1]{\Res\displaylimits_{#1}} 
\providecommand{\norm}[1]{\left\lVert#1\right\rVert}
%\providecommand{\norm}[1]{\lVert#1\rVert}
\providecommand{\mtx}[1]{\mathbf{#1}}
\providecommand{\mean}[1]{E\left[ #1 \right]}
\providecommand{\cond}[2]{#1\middle|#2}
\providecommand{\fourier}{\overset{\mathcal{F}}{ \rightleftharpoons}}
\newenvironment{amatrix}[1]{%
  \left(\begin{array}{@{}*{#1}{c}|c@{}}
}{%
  \end{array}\right)
}
%\providecommand{\hilbert}{\overset{\mathcal{H}}{ \rightleftharpoons}}
%\providecommand{\system}{\overset{\mathcal{H}}{ \longleftrightarrow}}
	%\newcommand{\solution}[2]{\textbf{Solution:}{#1}}
\newcommand{\solution}{\noindent \textbf{Solution: }}
\newcommand{\cosec}{\,\text{cosec}\,}
\providecommand{\dec}[2]{\ensuremath{\overset{#1}{\underset{#2}{\gtrless}}}}
\newcommand{\myvec}[1]{\ensuremath{\begin{pmatrix}#1\end{pmatrix}}}
\newcommand{\mydet}[1]{\ensuremath{\begin{vmatrix}#1\end{vmatrix}}}
\newcommand{\myaugvec}[2]{\ensuremath{\begin{amatrix}{#1}#2\end{amatrix}}}
\providecommand{\rank}{\text{rank}}
\providecommand{\pr}[1]{\ensuremath{\Pr\left(#1\right)}}
\providecommand{\qfunc}[1]{\ensuremath{Q\left(#1\right)}}
	\newcommand*{\permcomb}[4][0mu]{{{}^{#3}\mkern#1#2_{#4}}}
\newcommand*{\perm}[1][-3mu]{\permcomb[#1]{P}}
\newcommand*{\comb}[1][-1mu]{\permcomb[#1]{C}}
\providecommand{\qfunc}[1]{\ensuremath{Q\left(#1\right)}}
\providecommand{\gauss}[2]{\mathcal{N}\ensuremath{\left(#1,#2\right)}}
\providecommand{\diff}[2]{\ensuremath{\frac{d{#1}}{d{#2}}}}
\providecommand{\myceil}[1]{\left \lceil #1 \right \rceil }
\newcommand\figref{Fig.~\ref}
\newcommand\tabref{Table~\ref}
\newcommand{\sinc}{\,\text{sinc}\,}
\newcommand{\rect}{\,\text{rect}\,}
%%
%	%\newcommand{\solution}[2]{\textbf{Solution:}{#1}}
%\newcommand{\solution}{\noindent \textbf{Solution: }}
%\newcommand{\cosec}{\,\text{cosec}\,}
%\numberwithin{equation}{section}
%\numberwithin{equation}{subsection}
%\numberwithin{problem}{section}
%\numberwithin{definition}{section}
%\makeatletter
%\@addtoreset{figure}{problem}
%\makeatother

%\let\StandardTheFigure\thefigure
\let\vec\mathbf





\textbf{Question 1.1.7}
find the angles $\vec{A},\vec{B},\vec{C}$, given that 
\begin{align}
	\cos{A} \triangleq \frac{(\vec{B}-\vec{A})\top(\vec{C}-\vec{A})}{\norm{\vec{B}-\vec{A}}\norm{\vec{C}-\vec{A}}}
\end{align}
\fi
\textbf{Solution}:\\
From the given values of $\vec{A},\vec{B},\vec{C}$,\\
\begin{enumerate}
	\item Finding the value of angle A
\begin{align}
	\vec{B}-\vec{A} &=\myvec{-5\\7}
\end{align}
and 
\begin{align}
	\vec{C}-\vec{A}&= \myvec{-4\\-4}
\end{align}
also calculating the values of norms
\begin{align}
	\norm{\vec{B}-\vec{A}} &= \sqrt{74}\\
	\norm{\vec{C}-\vec{A}} &= \sqrt{32}
\end{align}
and by doing matrix multiplication we get,
\begin{align}
\begin{split}
	(\vec{B}-\vec{A})^{\top}(\vec{C}-\vec{A})&=\myvec{-5&7}\myvec{-4\\-4}\\
	&=-8
\end{split}
\end{align}
so 
\begin{align}
	\cos{A}&= \frac{-8}{\sqrt{74} \sqrt{32}}\\
	&= \frac{-1}{\sqrt{37}}\\
	\implies A&=\cos^{-1}{\frac{-1}{\sqrt{37}}}
\end{align}




\item Finding the value of angle B
\begin{align}
	\vec{C}-\vec{B} &=\myvec{1\\-11}
\end{align}
and 
\begin{align}
	\vec{A}-\vec{B}&= \myvec{5\\-7}
\end{align}
also calculating the values of norms
\begin{align}
	\norm{\vec{C}-\vec{B}} &= \sqrt{122}\\
	\norm{\vec{A}-\vec{B}} &= \sqrt{74}
\end{align}
and by doing matrix multiplication we get,
\begin{align}
\begin{split}
	(\vec{C}-\vec{B})^{\top}(\vec{A}-\vec{B})&=\myvec{1&-11}\myvec{5\\-7}\\
	&= 82
\end{split}
\end{align}
so 
\begin{align}
	\cos{B}&= \frac{82}{\sqrt{74} \sqrt{122}}\\
	&= \frac{41}{\sqrt{2257}}\\
	\implies B&=\cos^{-1}{\frac{41}{\sqrt{2257}}}
\end{align}



\item Finding the value of angle C
\begin{align}
	\vec{A}-\vec{C} &=\myvec{4\\4}
\end{align}
and 
\begin{align}
	\vec{B}-\vec{C}&= \myvec{-1\\11}
\end{align}
also calculating the values of norms
\begin{align}
	\norm{\vec{A}-\vec{C}} &= \sqrt{32}\\
	\norm{\vec{B}-\vec{C}} &= \sqrt{122}
\end{align}
and by doing matrix multiplication we get,
\begin{align}
\begin{split}
	(\vec{A}-\vec{C})^{\top}(\vec{B}-\vec{C})&=\myvec{4&4}\myvec{-1\\11}\\
	&=40
\end{split}
\end{align}
so 
\begin{align}
	\cos{C}&= \frac{40}{\sqrt{32} \sqrt{122}}\\
	&= \frac{5}{\sqrt{61}}\\
	\implies C&=\cos^{-1}{\frac{5}{\sqrt{61}}}
\end{align}

\end{enumerate}
