\subsection{Area of a Circle}
%
%
\renewcommand{\theequation}{\theenumi}
\begin{enumerate}[label=\arabic*.,ref=\thesubsection.\theenumi]
\numberwithin{equation}{enumi}
%
%
\item
	Using Fig. \ref{ch5_sin_theta}, show that 
	%
\begin{equation}
\label{ch5_sin_theta_eq}
\sin  \theta_1 = \sin \brak{\theta_1 + \theta_2}\cos \theta_2 - \cos\brak{\theta_1+\theta_2}\sin\theta_2
\end{equation}	
	%

\begin{figure}[!ht]
	\begin{center}
		
		%\includegraphics[width=\columnwidth]{./figs/ch5_sin_theta}
		%\vspace*{-10cm}
		\resizebox{\columnwidth}{!}{\begin{tikzpicture}
[scale =3,>=stealth,point/.style = {draw, circle, fill = black, inner sep = 1pt},]

\node (A) at (0,3)[point,label=above :$A$] {};
\node (B) at (3,0)[point,label=below :$B$] {};
\node (C) at (0,0)[point,label=below :$C$] {};
\node (D) at (0,1.5)[point,label=left :$D$] {};
\draw (A)--(B);
\draw (C)--(B);
\draw (A)--(C);
\draw (B)--(D);
\tkzMarkAngle[size=.4](A,B,D);
\tkzMarkAngle[size=.3](D,B,C);
\tkzMarkRightAngle[size=.15](A,C,B);

\node [above] at (1.6,1.5){$c$};
\node [below] at (1.6,0){$a$};
\node [below] at (1.6,1){$l$};
\node [above] at (-0.2,1.5){$b$};
\node [above] at (2.5,0){$\theta_2$};
\node [above] at (2.5,0.3){$\theta_1$};
\end{tikzpicture}}
	\end{center}
	\caption{$\sin2\theta = 2\sin\theta\cos\theta$}
	\label{ch5_sin_theta}	
\end{figure}
%

\solution The following equations can be obtained from the figure using the forumula for the area of a triangle
%
\begin{align}
ar \brak{\Delta ABC} &= \frac{1}{2}ac \sin\brak{\theta_1 + \theta_2} \\
&= ar \brak{\Delta BDC} + ar \brak{\Delta ADB} \\
&= \frac{1}{2}cl \sin{\theta_1} + \frac{1}{2}al \sin{\theta_2} \\ 
&= \frac{1}{2}ac \sin{\theta_1} \sec \theta_2 + \frac{1}{2}a^2 \tan{\theta_2} 
\end{align}
$\brak{\because
	l = a \sec \theta_2}$.  From the above,
\begin{align}
\Rightarrow \sin\brak{\theta_1 + \theta_2} &=  \sin{\theta_1} \sec \theta_2 + \frac{a}{c} \tan{\theta_2} \\
\Rightarrow \sin\brak{\theta_1 + \theta_2} &=  \sin{\theta_1} \sec \theta_2 + \cos\brak{\theta_1 + \theta_2} \tan{\theta_2} 
\end{align}
Multiplying both sides by $\cos \theta_2$,
\begin{align}
\Rightarrow \sin\brak{\theta_1 + \theta_2}\cos{\theta_2} &=  \sin{\theta_1}  + \cos\brak{\theta_1 + \theta_2} \sin\theta_2  
\end{align}
%
resulting in
\begin{equation}
\Rightarrow \sin \theta_1 = \sin\brak{\theta_1 + \theta_2}\cos{\theta_2} - \cos\brak{\theta_1 + \theta_2} \sin\theta_2 
\end{equation}
\item
	Prove the following identities 
	%
	\begin{enumerate}
\item 
\begin{equation}
		\label{ch5_sin_diff}
\sin\brak{\alpha - \beta} = \sin \alpha \cos \beta - \cos \alpha \sin \beta.
\end{equation}
\item 
\begin{equation}
\cos\brak{\alpha + \beta} = \cos \alpha \cos \beta - \sin \alpha \sin \beta.
		\label{ch5_cos_diff}
\end{equation}

	\end{enumerate}
	%

\solution In \eqref{ch5_sin_theta_eq}, let
%
\begin{equation}
\begin{split}
\theta_1 + \theta_2 &= \alpha \\
\theta_2 &=  \beta
\end{split}
\end{equation}
%
This gives \eqref{ch5_sin_diff}.  In \eqref{ch5_sin_diff}, replace $\alpha$ by 
%
$90{\degree} - \alpha$.  This results in
%
\begin{multline}
\sin\brak{90{\degree} - \alpha - \beta}
\\
=
\sin \brak{90{\degree} -\alpha} \cos \beta - \cos \brak{90{\degree} -\alpha} \sin \beta \\
\Rightarrow \cos\brak{\alpha + \beta} = \cos \alpha \cos \beta - \sin \alpha \sin \beta
\end{multline}
% 
\item
	Using \eqref{ch5_sin_theta_eq} and \eqref{ch5_cos_diff}, show that
\begin{align}
\label{ch5_sin_sum}
\sin\brak{\theta_1 + \theta_2} &= \sin\theta_1  \cos\theta_2 + \cos\theta_1\sin\theta_2
\\
\cos\brak{\theta_1 - \theta_2} &= \cos\theta_1  \cos\theta_2  \sin\theta_1\sin\theta_2
\label{ch5_cos_sum}
\end{align}

%
\solution From \eqref{ch5_sin_theta_eq},
%
\begin{align}
 \sin \brak{\theta_1 + \theta_2}\cos \theta_2 =\sin  \theta_1 +\cos\brak{\theta_1+\theta_2}\sin\theta_2 
\end{align}
%
Using \eqref{ch5_cos_diff} in the above,
%
\begin{multline}
\sin \brak{\theta_1 + \theta_2}\cos \theta_2 
=\sin  \theta_1 +\lbrak{\cos \theta_1\cos\theta_2 }
\\	
\rbrak{	- \sin \theta_1\sin\theta_2}\sin\theta_2 
\end{multline}
%
which can be expressed as
%
\begin{multline}
\sin \brak{\theta_1 + \theta_2}\cos \theta_2 
=\sin  \theta_1 +\cos \theta_1\cos\theta_2 \sin\theta_2 
\\	
	- \sin \theta_1\sin^2\theta_2
\end{multline}
%
Since
%
\begin{equation}
\sin^2\theta_2 = 1- \cos^2\theta_2, 
\end{equation}
%
we obtain
%
\begin{multline}
\sin \brak{\theta_1 + \theta_2}\cos \theta_2 
=\cos \theta_1\cos\theta_2 \sin\theta_2 
\\	
+ \sin \theta_1\cos^2\theta_2
\end{multline}
%
resulting in
%
\begin{equation}
\sin \brak{\theta_1 + \theta_2}
=\cos \theta_1 \sin\theta_2 
+ \sin \theta_1\cos\theta_2
\end{equation}
%
after factoring out $\cos \theta_2$.  Using a similar approach, \eqref{ch5_cos_sum} can also be proved.
%
\item
	Show that
	%
	\begin{equation}
	\label{eq:sin2theta}
	\sin 2\theta = 2 \sin\theta \cos\theta
	\end{equation}
	%
%
\item
	Show that
	%
	\begin{align}
	\label{eq:cos2theta_cos_sin}
	\cos 2\theta &= \cos^2\theta -\sin^2\theta 
\\
	\label{eq:cos2theta_sin_sq}
&= 1 - \sin^2\theta
\\
&=2\cos^2\theta -1
	\label{eq:cos2theta_cos_sq}
	\end{align}
	%
%\item The ratio of the perimeter of a circle to its diameter is $\pi$.
%\label{prob:circ_peri_dia}
%\item {\em Radian} is a another unit of the angle defined by
%%
%\begin{align}
%\pi \text{ radians} = 180 \degree
%\end{align}
%%
%\item
%	In Fig. \ref{ch5_polygon_def}, 6 congruent triangles are arranged in a circular fashion.  Such a figure is known as a regular hexagon.  In general, $n$ number of traingles can be arranged to form a regular polygon.
%\begin{figure}[!ht]
%	\begin{center}
%		
%		%\includegraphics[width=\columnwidth]{./figs/ch5_polygon_def}
%		%\vspace*{-10cm}
%		\resizebox{\columnwidth}{!}{\begin{tikzpicture}
[scale =2,>=stealth,point/.style = {draw, circle, fill = black, inner sep = 1pt},]

\node (O) at (0,0)[point,label=above :$O$] {};
\node (F) at (-3,0)[point,label=left :$F$] {};
\node (C) at (3,0)[point,label=right :$C$] {};
\node (E) at (-1.5,3)[point,label=above :$E$] {};
\node (D) at (1.5,3)[point,label=above :$D$] {};
\node (A) at (-1.5,-3)[point,label=below :$A$] {};
\node (B) at (1.5,-3)[point,label=below :$B$] {};
\draw (A)--(D);
\draw (B)--(E);
\draw (F)--(C);
\draw (A)--(F);
\draw (F)--(E);
\draw (E)--(D);
\draw (D)--(C);
\draw (C)--(B);
\draw (B)--(A);

\tkzMarkAngle[size=.2](F,O,A);
\tkzMarkAngle[size=.22](E,O,F);
\tkzMarkAngle[size=.26](D,O,E);
\tkzMarkAngle[size=.2](C,O,D);
\tkzMarkAngle[size=.22](B,O,C);
\tkzMarkAngle[size=.26](A,O,B);

\node [above] at (-0.8,-1.5){$r$};
\node [above] at (-1.5,0){$r$};
\node [above] at (-0.9,1.5){$r$};
\node [above] at (0.9,1.5){$r$};
\node [above] at (1.5,0){$r$};
\node [above] at (0.8,-1.5){$r$};
\node [above] at (0,-0.7){$\frac{2\pi}{6}$};

\end{tikzpicture}
}
%	\end{center}
%	\caption{Polygon Definition}
%	\label{ch5_polygon_def}	
%\end{figure}
%%
%\item
%The angle formed by each of the congruent triangles at the centre of a regular polygon of $n$ sides is $\frac{2\pi}{n} = \frac{2\pi}{n}$ rad.
%%
%\item 	The triangle that forms a polygon of $n$ sides is given in Fig. \ref{ch5_polygon_area}. Show that 
%%
%\begin{equation}
%BC = 2r \sin\frac{\pi}{n}
%\label{eq:circ_chord_len}
%\end{equation}
%%
%
%\begin{figure}[!ht]
%	\begin{center}
%		
%		%\includegraphics[width=\columnwidth]{./figs/ch5_polygon_area}
%		%\vspace*{-10cm}
%		\resizebox{\columnwidth}{!}{\begin{tikzpicture}
[scale =2,>=stealth,point/.style = {draw, circle, fill = black, inner sep = 1pt},]

\node (A) at (0,3)[point,label=above :$A$] {};
\node (B) at (-3,0)[point,label=below :$B$] {};
\node (C) at (3,0)[point,label=below :$C$] {};
\draw (A)--(B);
\draw (C)--(B);
\draw (A)--(C);

\tkzMarkAngle[size=.2](B,A,C);
\node [above] at (0.05,2.5){$\frac{2\pi}{n}$};
\node [above] at (-1.6,1.5){$r$};
\node [above] at (1.6,1.5){$r$};
\end{tikzpicture}
}
%	\end{center}
%	\caption{Triangle that forms a polygon}
%	\label{ch5_polygon_area}	
%\end{figure}
%%
%%\\
%\solution Using cosine formula, 
%%
%%
%\begin{align}
%BC^2 &= 2r^2-2r^2 \cos\frac{2\pi}{n}
%\\
%\implies BC^2 &= 2r^2\brak{1-\cos\frac{2\pi}{n}}
%&= 4r^2\sin^2\frac{\pi}{n}
%\end{align}
%%
%upon substituting from 	\eqref{eq:cos2theta_sin_sq}.  Taking the square root results in \eqref{eq:circ_chord_len}
%%
%\item
%Show that the perimeter of a regular polygon is given by 
%%
%\begin{equation}
%\label{eq:peri_poly_n}
%2rn \sin\frac{\pi}{n}
%\end{equation}
%%
%\item
%Show that the area of a regular polygon is given by 
%%
%\begin{equation}
%\frac{n}{2}r^{2}\sin\frac{2\pi}{n}
%\end{equation}
%%
%\solution  From Fig. 	\ref{ch5_polygon_area}	
%
%%
%\begin{equation}
%\begin{split}
%ar\brak{polygon} &= n \times ar\brak{\Delta ABC} \\
%&= \frac{n}{2}r^{2}\sin\frac{2\pi}{n}
%\end{split}
%\end{equation}
%%
%\item
%	Using Fig. \ref{ch5_circle_squeeze}, show that
%%
%\begin{equation}
%\label{ch5_circle_squeeze_eq}
%\frac{n}{2}r^{2}\sin\frac{2\pi}{n} < \text{ area of circle } < nr^{2}\tan\frac{\pi}{n}
%\end{equation}
%%
%The portion of the circle visible in Fig. \ref{ch5_circle_squeeze} is defined to be a sector of the circle.
%
%\begin{figure}[!ht]
%	\begin{center}
%		
%		%\includegraphics[width=\columnwidth]{./figs/ch5_circle_squeeze}
%		%\vspace*{-10cm}
%		\resizebox{\columnwidth}{!}{\begin{tikzpicture}
[scale =2,>=stealth,point/.style = {draw, circle, fill = black, inner sep = 1pt},]

\def\rad{3}
\coordinate [point, label={right: $O$ }] (O) at (0, 0);
\draw (O) circle (\rad);
\node (P) at (0,-3)[point,label=below :$P$] {};
\node (A) at (-2,-2.22)[point,label= left :$A$] {};
\node (B) at (2,-2.22)[point,label=right :$B$] {};
\node (C) at (-2.7,-3)[point,label=below :$C$] {};
\node (D) at (2.7,-3)[point,label=below :$D$] {};
\draw (A)--(B);
\draw (O)--(B);
\draw (A)--(O);
\draw (O)--(P);
\draw (C)--(P);
\draw (D)--(P);

\draw [thick,dashed](A)--(C);
\draw [thick,dashed](B)--(D);
\tkzMarkRightAngle[size=0.2](O,P,D)
\tkzMarkAngle[size=0.2](C,O,P)
\tkzMarkAngle[size=0.3](P,O,D)
\node [above] at (-1.4,-1.5){$r$};
\node [above right] at (0,-1.5){$r$};
\node [above] at (1.4,-1.5){$r$};
\node [above right] at (0.4,-0.5){$\theta=\frac{2\pi}{n}$};
\end{tikzpicture}
}
%	\end{center}
%	\caption{Circle Area in between Area of Two Polygons}
%	\label{ch5_circle_squeeze}	
%\end{figure}
%%
%
%\solution Note that the circle is squeezed between the inner and outer regular polygons.  As we can see from Fig. \ref{ch5_circle_squeeze}, the area of the circle should be in between the areas of the inner and outer polygons.  Since
%%
%\begin{align}
%ar \brak{\Delta OAB} &= \frac{1}{2}r^{2}\sin\frac{2\pi}{n} \\
%ar \brak{\Delta OPQ} &= 2 \times \frac{1}{2} \times r \tan\frac{2\pi/n}{2} \times r \\
%&= r^{2}\tan\frac{\pi}{n},
%\end{align}
%%
%we obtain \eqref{ch5_circle_squeeze_eq}.
%%
%\item
%Show that
%	%
%	\begin{equation}
%	\label{ch5_circle_squeeze_simple}
%\cos^2\frac{\pi}{n} < \frac{\text{ area of circle }}{nr^{2}\tan\frac{\pi}{n}} < 1	\end{equation}
%	%
%
%\solution From \eqref{ch5_circle_squeeze_eq} and \eqref{eq:sin2theta},
%	%
%{\small
%	\begin{align}
%	\frac{n}{2}r^{2}\sin\frac{2\pi}{n} < \text{ area of circle } 
%%	\\
%	< nr^{2}\tan\frac{\pi}{n} 
%	\\
%\Rightarrow 	
%	{n}r^{2}\sin\frac{\pi}{n}\cos\frac{\pi}{n} < \text{ area of circle } 
%%	\\
%	< nr^{2}\tan\frac{\pi}{n} 
%	\end{align}
%%
%}
%which yields 	\eqref{ch5_circle_squeeze_simple} upon making use of the fact that 
%%
%\begin{align}
%\frac{\sin \theta}{\cos \theta} = \tan \theta
%\end{align}
%%
%
%%
%\item
%	Show that 
%	\begin{equation}
%	\label{ch5_cos_zero}
%	\cos 0^{\degree} = 1
%	\end{equation}
%
%\solution Follows from the fact that $\cos 0 \degree = \sin \brak{90\degree -0\degree} = \sin \brak{90\degree }=1$ using \eqref{eq:tri_90-ang}.
%
%
%%From \eqref{ch1_trig_defs}, $\theta \to 0\degree \implies a \to 0 \implies \sin \theta $ and %
%
%\item
%	Show that 
%	\begin{equation}
%	\label{ch5_sin_zero}
%	\sin 0^{\degree} = 0
%	\end{equation}
%%\solution Follows from the fact that $\sin 0 = 0$ and \eqref{eq:tri_sin_cos_id}.
%
%\item
%	Show that for large values of $n$
%	%
%	\begin{equation}
%\label{eq:cos_zero_lim}
%	%
%\cos^2\frac{\pi}{n} = 1
%%
%	\end{equation}	
%	% 
%
%%
%\solution  As $n \to \infty, \frac{\pi}{n} \to 0$. From \eqref{ch5_cos_zero}, this yields \eqref{eq:cos_zero_lim}.
%
%%
%\item  \eqref{eq:cos_zero_lim} is a {\em limit} and 
%	 expressed as 
%%
%\begin{equation}
%\label{eq:cos_zero_lim_def}
%\lim_{n \rightarrow \infty}\cos^2\frac{\pi}{n} = 1
%\end{equation}
%%	
%
%\item
%	Show that 
%	%
%\begin{equation}
%\label{ch5_circle_squeeze_tan_n}
%\text{ area of circle } = r^2\lim_{n \rightarrow \infty}
%{n\tan\frac{\pi}{n}} 
%	%
%\end{equation}	
%	% 
%\solution From \eqref{ch5_circle_squeeze_simple} and \eqref{eq:cos_zero_lim_def}, 
%	%
%	\begin{align}
%%	\label{ch5_circle_squeeze_simple}
%\lim_{n\to \infty}\cos^2\frac{\pi}{n} < \lim_{n\to \infty} \frac{\text{ area of circle }}{nr^{2}\tan\frac{\pi}{n}} < 1	
%\\
%1 = \lim_{n\to \infty} \frac{\text{ area of circle }}{nr^{2}\tan\frac{\pi}{n}} < 1	
%\end{align}
%	%
%resulting in \eqref{ch5_circle_squeeze_tan_n}.
%%
%
%%
%\item Show that 
%
%	\begin{equation}
%\label{eq:peri_poly_tan_n}
%	\pi = \lim_{n \rightarrow \infty}
%	{n\tan\frac{\pi}{n}}
%	\end{equation}
%\solution From \eqref{prob:circ_peri_dia} and \eqref{eq:peri_poly_n}, the perimeter of the circle is 
%%
%\begin{align}
%\label{eq:peri_poly_sin_n}
%\lim_{n\to \infty}2rn \sin\frac{\pi}{n} &= 2\pi r
%\implies \lim_{n\to \infty}n \sin\frac{\pi}{n} &= \pi 
%\end{align}
%%
%Also, from Fig. \eqref{ch5_circle_squeeze}, using the fact that the inner and outer polygons converge into a circle for large $n$,
%\begin{align}
%%\label{eq:peri_poly_n}
%\lim_{n\to \infty} nCD -nAB &= 0
%\\
%\implies \lim_{n\to \infty} 2r n\tan\frac{\pi}{n}-2r n\sin\frac{\pi}{n} &= 0
%\end{align}
%%
%from which, we obtain \eqref{eq:peri_poly_tan_n}
%by substituting from \eqref{eq:peri_poly_sin_n}.
%
%
%\item Show that the area of a circle is $\pi r^2$.
%\\
%\solution Use \eqref{eq:peri_poly_tan_n} in \eqref{ch5_circle_squeeze_tan_n}.
%
%%\item
%%	The radian is a unit of angle defined by
%%\begin{equation}
%%	1 \text{ radian} = \frac{2\pi}{2\pi}
%%\end{equation}
%
%%
%%\item
%%	Show that the circumference of a circle is $2 \pi r$.
%\item
%	Show that
%	\begin{equation}
%	\lim_{\theta \rightarrow 0} \frac{\sin\theta}{\theta} =
%	\lim_{\theta \rightarrow 0} \frac{\tan\theta}{\theta} = 1
%	\end{equation}
%
%\item
%	Show that the area of a sector with angle $\theta$ in radians is $\frac{1}{2}r^2\theta$.
%

\end{enumerate}
