%%
\renewcommand{\theequation}{\theenumi}
\begin{enumerate}[label=\thesection.\arabic*.,ref=\thesection.\theenumi]
\numberwithin{equation}{enumi}

\item In Fig. \ref{fig:tri_ccentre}, points $\vec{A}, \vec{B}, \vec{C}$  are at a distance $R$ from $\vec{O}$.  Trace all such points. The locus of such points is defined as a {\em circle}.
%
	\iffalse
\\
\solution This is done by the following python code
%
\begin{lstlisting}
codes/circle/tri_ccircle.py
\end{lstlisting}
%
and the equivalent latex-tikz code to draw Fig. \ref{fig:tri_ccircle} is
%
\begin{lstlisting}
figs/triangle/tri_ccircle.tex
\end{lstlisting}
\fi
\begin{figure}[!ht]
	\begin{center}
		
		\resizebox{\columnwidth}{!}{%Code by GVV Sharma
%December 9, 2019
%released under GNU GPL
%Locating the circumcentre

\begin{tikzpicture}
[scale=2,>=stealth,point/.style={draw,circle,fill = black,inner sep=0.5pt},]

%Triangle sides
\def\a{5}
\def\b{6}
\def\c{4}
\def\R{3.023715784073818}
 
%Coordinates of A
%\def\p{{\a^2+\c^2-\b^2}/{(2*\a)}}
\def\p{0.5}
\def\q{{sqrt(\c^2-\p^2)}}

% Vertices
\node (A) at (\p,\q)[point,label=above right:$A$] {};
\node (B) at (0, 0)[point,label=below left:$B$] {};
\node (C) at (\a, 0)[point,label=below right:$C$] {};

% Mid points
\node (D) at ($(B)!0.5!(C)$)[point,label=below:$D$] {};
\node (E) at ($(C)!0.5!(A)$)[point,label=right:$E$] {};
\node (F) at ($(B)!0.5!(A)$)[point,label=left:$F$] {};

%Circumcentre

\node (O) at (2.5,1.70084013)[point,label=right:$O$] {};

%Drawing triangle ABC
\draw (A) -- node[above left, yshift=2mm] {$\textrm{c}$} (B) -- node[below right, xshift = 2mm] {$\textrm{a}$} (C) -- node[above,yshift=2mm] {$\textrm{b}$} (A);
%Drawing OA, OB, OC
\draw (O) -- node[left] {$\textrm{R}$} (A);
\draw (O) -- node[below] {$\textrm{R}$} (B);
\draw (O) -- node[below] {$\textrm{R}$} (C);

%Drawing OD, OE, OF
\draw (O) --  (D);
\draw (O) --  (E);
\draw (O) --  (F);


%Drawing circumcircle
\draw (O) circle (\R);

\tkzMarkRightAngle[fill=blue!20,size=.2](O,D,C)
\tkzMarkRightAngle[fill=blue!20,size=.2](O,E,A)
\tkzMarkRightAngle[fill=blue!20,size=.2](O,F,B)

\end{tikzpicture}
}
	\end{center}
	\caption{Circumcircle of $\triangle ABC$}
	\label{fig:tri_ccircle}	
\end{figure}
%
\item Line segements joining any two points on the circle are defined to be {\em chords}. The sides of $\triangle ABC$ are chords of the circle in 	\eqref{fig:tri_ccircle}	

\item From  \eqref{prob:tri_perp_bisect}, it is established that the line segment joining the centre of a circle to the mid point of a chord bisects the chord.
\item From \eqref{prob:tri_ccentre_subtend}, it is clear that the angle subtended by a chord at the centre of the circle is twice the angle subtended at any point on the circle.
\label{them:tri_ccentre_subtend}
%

\end{enumerate}

