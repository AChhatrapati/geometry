%%
\renewcommand{\theequation}{\theenumi}
\begin{enumerate}[label=\arabic*.,ref=\thesubsection.\theenumi]
\numberwithin{equation}{enumi}

\item In Fig. \ref{fig:tri_icentre}, points $\vec{D}, \vec{E}, \vec{F}$  are at a distance $r$ from $\vec{I}$.  The circle with centre $\vec{I}$ through these points is known as the {\em incircle}. Draw the incircle of $\triangle ABC$.
%
\\
\solution This is done by the following python code
%
\begin{lstlisting}
codes/circle/tri_icircle.py
\end{lstlisting}
%
and the equivalent latex-tikz code to draw Fig. \ref{fig:tri_icircle} is
%
\begin{lstlisting}
figs/triangle/tri_icircle.tex
\end{lstlisting}

\begin{figure}[!ht]
	\begin{center}
		
		\resizebox{\columnwidth}{!}{%Code by GVV Sharma
%December 10, 2019
%released under GNU GPL
%Drawing the incircle

\begin{tikzpicture}
[scale=2,>=stealth,point/.style={draw,circle,fill = black,inner sep=0.5pt},]

%Triangle sides
\def\a{5}
\def\b{6}
\def\c{4}

%Inradius
\def\r{1.3228756555322954}
 
%Coordinates of A
%\def\p{{\a^2+\c^2-\b^2}/{(2*\a)}}
\def\p{0.5}
\def\q{{sqrt(\c^2-\p^2)}}

%Labeling points
\node (A) at (\p,\q)[point,label=above right:$A$] {};
\node (B) at (0, 0)[point,label=below left:$B$] {};
\node (C) at (\a, 0)[point,label=below right:$C$] {};

%Circumcentre

\node (I) at (1.5,1.32287566)[point,label=right:$I$] {};
\node (D) at (1.5,0)[point,label=below:$D$] {};
\node (E) at (2.375,2.3150324)[point,label=above right:$E$] {};
\node (F) at (0.1875,1.48823511)[point,label=left:$F$] {};

%Drawing triangle ABC
\draw (A) -- node[left] {$\textrm{c}$} (B) -- node[below] {$\textrm{a}$} (C) -- node[above,yshift=2mm] {$\textrm{b}$} (A);
%Drawing OA, OB, OC
\draw (I) --  (A);
\draw (I) --  (B);
\draw (I) --  (C);

%Drawing OD, OE, OF
\draw (I) -- node[right] {$\textrm{r}$} (D);
\draw (I) -- node[below] {$\textrm{r}$} (E);
\draw (I) -- node[below] {$\textrm{r}$} (F);

%Drawing Incircle
\draw (I) circle (\r);



\tkzMarkAngle[fill=green!60,size=.3](I,B,F)
\tkzMarkAngle[fill=green!40,size=.3](D,B,I)
%
%
\tkzMarkAngle[fill=red!60,size=.3](F,A,I)
\tkzMarkAngle[fill=red!40,size=.3](I,A,E)


\tkzMarkAngle[fill=orange!60](E,C,I)
\tkzMarkAngle[fill=orange!40](I,C,D)
%
\tkzMarkRightAngle[fill=blue!20,size=.3](A,E,I)
\tkzMarkRightAngle[fill=blue!20,size=.3](A,F,I)
\tkzMarkRightAngle[fill=blue!20,size=.3](I,D,C)

%Labeling x,y,z
\node (x1) at ($(B)!0.5!(D)$)[label=below:$x$] {};
\node (x2) at ($(B)!0.5!(F)$)[label=left:$x$] {};
\node (y1) at ($(C)!0.5!(D)$)[label=below:$y$] {};
\node (y2) at ($(C)!0.5!(E)$)[label=right:$y$] {};
\node (z1) at ($(A)!0.5!(E)$)[label=right:$z$] {};
\node (z2) at ($(A)!0.5!(F)$)[label=left:$z$] {};


\end{tikzpicture}
}
	\end{center}
	\caption{Circumcircle of $\triangle ABC$}
	\label{fig:tri_icircle}	
\end{figure}
%
\item Sides $AB, BC$ and $CA$ touch the circle at exactly one point.  Such lines are known as {\em tangents} to the circle.
\item Tangents to the circle are perpendicular to the radius at the point of contact.
\item From \eqref{eq:tri_icentre_baudhd}, it is  obvious that tangents to the circle from a given point are equal.
%

\end{enumerate}

