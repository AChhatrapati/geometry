%\subsection{Properties}
%
\renewcommand{\theequation}{\theenumi}
\begin{enumerate}[label=\thesection.\arabic*.,ref=\thesection.\theenumi]
\numberwithin{equation}{enumi}
\numberwithin{figure}{enumi}
%
\item Find the equation of the line $BC$.
%
\begin{figure}[!ht]
	\begin{center}
		\resizebox{\columnwidth}{!}{%Code by GVV Sharma
%December 15, 2019
%released under GNU GPL
%Drawing the altitude of a triangle

\begin{tikzpicture}
[scale=2,>=stealth,point/.style={draw,circle,fill = black,inner sep=0.5pt},]

%Labeling Vertices
\node (A) at (2.25,3.30718914)[point,label=above:$A$] {};
\node (B) at (0, 0)[point,label=below left:$B$] {};
\node (C) at (6, 0)[point,label=below right:$C$] {};

%Foot of perpendicular

\node (D) at (2.25, 0)[point,label=below:$D$] {};

%Drawing triangle ABC
\draw (A) -- node[left] {$\textrm{c}$} (B) -- node[below] {$\textrm{a}$} (C) -- node[above,xshift=2mm] {$\textrm{b}$} (A);

%Drawing altitude AD
\draw (A) -- (D);

\tkzMarkRightAngle[fill=blue!20,size=.2](A,D,B)

\end{tikzpicture}
}
	\end{center}
	\caption{Drawing the altitude}
	\label{fig:tri_alt}	
\end{figure}
\\
\solution Let $\vec{x}$ be any point on $BC$.  Using section formula, for some $k$, 
%
\begin{align}
\vec{x} &= \frac{k\vec{C}+\vec{B}}{k+1} = \frac{\brak{k+1}\vec{C}+\brak{\vec{B}-\vec{C}}}{k+1}
\\
\implies \vec{x} &= \vec{C} + \lambda \vec{m}
\label{eq:line_dir_pt}
\end{align}
%
where 
%
\begin{align}
\vec{m} 
 = \frac{\vec{B}-\vec{C}}{k+1} \equiv \vec{B}-\vec{C}
\label{eq:line_dir_pt-alt}
\end{align}
%
\item The {\em normal vector} to $\vec{m}$ is defined as
%
\begin{align}
\label{eq:dir_normal_orth}
\vec{n}^{\top}\vec{m} &= 0
\end{align}
%
%
\begin{align}
\vec{n} &= \myvec{0 & -1 \\ 1 & 0}\vec{m}
\end{align}
%
\item 
From \eqref{eq:dir_normal_orth} and \eqref{eq:line_dir_pt}, 
it can be verified that 
%
\begin{align}
\vec{n}^{\top}\vec{x} &= \vec{n}^{\top}\vec{C} + \lambda \vec{n}^{\top}\vec{m}
\\
\implies \vec{n}^{\top}\vec{x} &= \vec{n}^{\top}\vec{C}
\label{eq:line_normal_pt}
\end{align}
%
\eqref{eq:line_normal_pt} is defined to be the {\em normal form} of the line $BC$.  
%
\item 
	In \figref{fig:tri_alt_h},	
$AD \perp BC$ and $BE \perp AC$ are defined to be the altitudes of $\triangle ABC$. 
\item Let $\vec{H}$ be the intersection of the altitudes $AD$ and $BE$ as shown in Fig. \ref{fig:tri_alt_h}.  $CH$ is extended to meet $AB$ at $\vec{F}$.  Show that $CF \perp AB$.
%
\begin{figure}[!ht]
	\begin{center}
		\resizebox{\columnwidth}{!}{%Code by GVV Sharma
%December 15, 2019
%released under GNU GPL
%Orthocentre of a triangle

\begin{tikzpicture}
[scale=2,>=stealth,point/.style={draw,circle,fill = black,inner sep=0.5pt},]

%Labeling Vertices
\node (A) at (2.25,3.30718914)[point,label=above:$A$] {};
\node (B) at (0, 0)[point,label=below left:$B$] {};
\node (C) at (6, 0)[point,label=below right:$C$] {};

%Foot of perpendicular

\node (D) at (2.25, 0)[point,label=below:$D$] {};
\node (E) at (2.625, 2.97647022)[point,label=below:$E$] {};
\node (F) at (1.8984375, 2.79044084)[point,label=below:$F$] {};
\node (H) at (2.25,2.55126019)[point,label=below:$D$] {};

%Drawing triangle ABC
\draw (A) -- node[left] {$\textrm{c}$} (B) -- node[below] {$\textrm{a}$} (C) -- node[above,xshift=2mm] {$\textrm{b}$} (A);

%Drawing altitudes
\draw (A) -- (D);
\draw (B) -- (E);
\draw [dashed] (C) -- (F);

\tkzMarkRightAngle[fill=blue!20,size=.2](A,D,B)
\tkzMarkRightAngle[fill=blue!20,size=.2](B,E,A)

\end{tikzpicture}
}
	\end{center}
	\caption{Altitudes of a triangle meet at the orthocentre $H$}
	\label{fig:tri_alt_h}	
\end{figure}
%
\\
\solution 
From \eqref{eq:line_dir_pt-alt}
\eqref{eq:dir_normal_orth},
  \eqref{eq:dot2d-orth}
and 
\eqref{eq:line_normal_pt},
the equations of $AD$ and $BE$ are 
%
\begin{align}
\brak{\vec{B}-\vec{C}}^{\top}\brak{\vec{x}-\vec{A}} &= 0  
\\
\brak{\vec{C}-\vec{A}}^{\top}\brak{\vec{x}-\vec{B}} &= 0  
\end{align}
%
 $\because \vec{H}$ lies on both $AD$ and $BE$, it satisfies the above equations, and 
%
\begin{align}
\brak{\vec{B}-\vec{C}}^{\top}\brak{\vec{H}-\vec{A}} &= 0  
\\
\brak{\vec{C}-\vec{A}}^{\top}\brak{\vec{H}-\vec{B}} &= 0  
\end{align}
%
Adding both the above and simplifying, 
%
\begin{align}
\brak{\vec{B}-\vec{A}}^{\top}\brak{\vec{H}-\vec{C}} &= 0  
\end{align}
%
$\implies CH \perp AB$ 
from   \eqref{eq:dot2d-orth}, or $CF \perp AB$.  
%
\iffalse
The python code for  Fig. \ref{fig:tri_alt_h} is
\begin{lstlisting}
codes/triangle/tri_alt_h.py
\end{lstlisting}
%
and the equivalent latex-tikz code is
%
\begin{lstlisting}
figs/triangle/tri_alt_h.tex
\end{lstlisting}
\fi
\item Altitudes of a $\triangle$ meet at the {\em orthocentre} $H$.
%
\end{enumerate}
