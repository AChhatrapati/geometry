%%
%\subsection{Perpendicular Bisectors}
\renewcommand{\theequation}{\theenumi}
\begin{enumerate}[label=\thesection.\arabic*.,ref=\thesection.\theenumi]
\numberwithin{equation}{enumi}
\item In 
	\figref{fig:tri-isosc},	
\begin{figure}[!ht]
	\begin{center}
		\resizebox{\columnwidth}{!}{%Code by GVV Sharma
%July 6, 2023
%Revised July 7, 2023
%released under GNU GPL
%The Isosceles Triangle

\begin{tikzpicture}
[scale=2,>=stealth,point/.style={draw,circle,fill = black,inner sep=0.5pt},]

%Triangle sides
\def\a{5}
\def\b{6}
\def\c{4}
 
%Coordinates of A
%\def\p{{\a^2+\c^2-\b^2}/{(2*\a)}}
\def\p{0.5}
\def\q{{sqrt(\c^2-\p^2)}}

%Labeling points
%\node (A) at (\p,\q)[point,label=above right:$A$] {};
\node (B) at (0, 0)[point,label=below left:$B$] {};
\node (C) at (\a, 0)[point,label=below right:$C$] {};

%Circumcentre

\node (O) at (2.5,1.70084013)[point,label=above right:$O$] {};

%Drawing triangle OBC
%\draw (A) -- node[left] {$\textrm{c}$} (B) -- node[below] {$\textrm{a}$} (C) -- node[above,yshift=2mm] {$\textrm{b}$} (A);
%Drawing OA, OB, OC
%\draw (O) -- node[left] {$\textrm{R}$} (A);
\draw (O) -- node[below] {${R}$} (B);
\draw (O) -- node[below] {${R}$} (C);
\draw (B) -- node[below] {${a}$} (C);

\tkzMarkAngle[fill=blue!50,size=.3](C,B,O)
\tkzMarkAngle[fill=blue!50,size=.3](O,C,B)


\tkzMarkAngle[fill=red!10](B,O,C)
\tkzLabelAngle[pos=0.3](B,O,C){$\theta$}
%\tkzMarkAngle[fill=red!10](A,C,O)

\iffalse
\tkzMarkAngle[fill=orange!50,size=.3](B,A,O)
\tkzMarkAngle[fill=orange!50,size=.3](O,B,A)

\tkzLabelAngle[pos=0.5](O,C,B){$\theta_1$}
\tkzLabelAngle[pos=0.5](O,B,C){$\theta_1$}
\tkzLabelAngle[pos=0.5](O,A,B){$\theta_2$}
\tkzLabelAngle[pos=0.5](O,B,A){$\theta_2$}
\tkzLabelAngle[pos=1.5](O,A,C){$\theta_3$}
\tkzLabelAngle[pos=1.5](O,C,A){$\theta_3$}
\fi

\end{tikzpicture}
}
	\end{center}
	\caption{Isosceles Triangle}
	\label{fig:tri-isosc}	
\end{figure}
\begin{align}
	OB = OC=R
\end{align}
Such a triangle is known as an isosceles triangle.  Show that
\begin{align}
	\angle B = \angle C
\end{align}
\solution 
Using
\eqref{eq:tri_sin_form},
\begin{align}
	\frac{\sin B}{R} &= \frac{\sin C}{R}
	\\
\implies	{\sin B} &= {\sin C}
\\
	\text{or, } \angle B &= \angle C.
\end{align}
\item In 
	\figref{fig:tri-isosc},	
	show that 
  \begin{align}
	  a = 2R \sin\frac{ \theta }{2}
\label{eq:crad_cos2a}
  \end{align}
		\solution In $\triangle OBC$,  using the cosine formula from
\eqref{eq:tri_cos_form},
\begin{align}
	\cos \theta &= \frac{R^2+R^2 - a^2}{2R^2} = 1 -\frac{a^2}{2R^2}
	\\
	\implies \frac{a^2}{2R^2}&= 2\sin^2\frac{\theta}{2}
\end{align}
yielding 
\eqref{eq:crad_cos2a}.
\item In
	\figref{fig:tri_ccircle-ang},
show that 
\begin{align}
\label{eq:tri_crad_R}
\frac{a}{\sin A} = \frac{b}{\sin B} = \frac{c}{\sin C} = 2R.
\end{align}
%
%
\solution
From 
\eqref{eq:ang-subtend-ccentre}
and 
\eqref{eq:crad_cos2a}
  \begin{align}
	  a = 2R \sin A
  \end{align}


\end{enumerate}
\iffalse
\begin{figure}[!ht]
	\begin{center}
		
		\resizebox{\columnwidth}{!}{%Code by GVV Sharma
%December 9, 2019
%released under GNU GPL
%Locating the circumcentre

\begin{tikzpicture}
[scale=2,>=stealth,point/.style={draw,circle,fill = black,inner sep=0.5pt},]

%Triangle sides
\def\a{5}
\def\b{6}
\def\c{4}
 
%Coordinates of A
%\def\p{{\a^2+\c^2-\b^2}/{(2*\a)}}
\def\p{0.5}
\def\q{{sqrt(\c^2-\p^2)}}

%Labeling points
\node (A) at (\p,\q)[point,label=above right:$A$] {};
\node (B) at (0, 0)[point,label=below left:$B$] {};
\node (C) at (\a, 0)[point,label=below right:$C$] {};

%Circumcentre

\node (O) at (2.5,1.70084013)[point,label=above right:$O$] {};

%Drawing triangle ABC
\draw (A) -- node[left] {$\textrm{c}$} (B) -- node[below] {$\textrm{a}$} (C) -- node[above,yshift=2mm] {$\textrm{b}$} (A);
%Drawing OA, OB, OC
\draw (O) -- node[left] {$\textrm{R}$} (A);
\draw (O) -- node[below] {$\textrm{R}$} (B);
\draw (O) -- node[below] {$\textrm{R}$} (C);

\tkzMarkAngle[fill=blue!50,size=.3](C,B,O)
\tkzMarkAngle[fill=blue!50,size=.3](O,C,B)


%\tkzMarkAngle[fill=red!10](O,A,C)
%\tkzMarkAngle[fill=red!10](A,C,O)


\tkzMarkAngle[fill=orange!50,size=.3](B,A,O)
\tkzMarkAngle[fill=orange!50,size=.3](O,B,A)

\tkzLabelAngle[pos=0.5](O,C,B){$\theta_1$}
\tkzLabelAngle[pos=0.5](O,B,C){$\theta_1$}
\tkzLabelAngle[pos=0.5](O,A,B){$\theta_2$}
\tkzLabelAngle[pos=0.5](O,B,A){$\theta_2$}
\tkzLabelAngle[pos=1.5](O,A,C){$\theta_3$}
\tkzLabelAngle[pos=1.5](O,C,A){$\theta_3$}

\end{tikzpicture}
}
	\end{center}
	\caption{Circumcentre $O$ of $\triangle ABC$}
	\label{fig:tri_ccentre}	
\end{figure}
\fi
  \iffalse
Using the sine formula, 
\begin{align}
\frac{\sin 2A}{a} &= \frac{\sin \theta_1}{R} = \frac{\sin\brak{90\degree- A}}{R}
\\
\implies \sin 2A &= \frac{a\cos A}{R}
\label{eq:crad_sin2a}
\end{align}
%
from \eqref{eq:tri_ccentre_A1} and \eqref{eq:tri_baudh_comp}.	Using \eqref{eq:tri_sin_cos_id}, 
\begin{align}
\cos^2 2A + \sin^2 2A&= 1
\\
\implies \brak{1 -\frac{a^2}{2R^2}}^2 + \brak{\frac{a\cos A}{R}}^2 &= 1
\end{align}
%
upon substituting from \eqref{eq:crad_cos2a}  and \eqref{eq:crad_sin2a}.  Letting
%
\begin{align}
\label{eq:tri_crad_x}
x = \brak{\frac{a}{R}}^2,
\end{align}
%
in the previous equation yields
%
\begin{align}
 \brak{1 -\frac{x}{2}}^2 + x\cos^2 A&= 1
\\
\implies 1 - \frac{x^2}{4} -x + x\cos^2 A&= 1
\\
\implies x\brak{1-\cos^2 A} - \frac{x^2}{4} &= 0
\end{align}
%
From \eqref{eq:tri_sin_cos_id}, the above equation can be expressed as
%
\begin{align}
x\sin^2 A - \frac{x^2}{4} &= 0
\\
\implies x\brak{\sin^2 A - \frac{x}{4}} &= 0
\\
\text{or, } \frac{x}{4} - \sin^2 A &=0
\end{align}
%
$\because x \ne 0$.  Thus, substituting from \eqref{eq:tri_crad_x},
\begin{align}
x = \brak{\frac{a}{R}}^2 &= 4 \sin^2 A 
\\
\implies \frac{a}{R} &= 2\sin A,
\\
\text{or, }\quad \frac{a}{\sin A} = 2R
%\label{eq:circ_chord_len}
\end{align}
%
\item Show that 
\label{eq:cos2x}
\begin{align}
\cos 2A &= 1 -2\sin^2 A = 2\cos^2 A - 1 
\\
&= \cos^2 A - \sin^2A \quad \text{ and }
\\
\sin 2A &= 2 \sin A \cos A
\label{eq:sin2x}
\end{align}
\item Find $R$.
\\
\solution From \eqref{eq:tri_area_sin}, 
\begin{align}
ar\brak{\triangle ABC} = \frac{1}{2}bc \sin A = \frac{abc}{4R}&
\\
\implies R = \frac{abc}{4s\sqrt{\brak{s-a}\brak{s-b}\brak{s-c}}}&
\end{align}
%
upon substituting from \eqref{eq:tri_crad_R} and using Hero's formula.
%
\item Show that
%
\begin{align}
\label{eq:circ_area_chord}
ar\brak{\triangle OBC} = \frac{1}{2}R^2\sin 2A
\end{align}
%
\item Find the circumradius of $\triangle ABC$ for $a = 5, b = 6, c = 4$.
%
\\
\solution The following python code calculates the circumradius
\begin{lstlisting}
codes/circle/tri_cradius.py
\end{lstlisting}
\fi
