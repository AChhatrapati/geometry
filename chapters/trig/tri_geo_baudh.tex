Use Fig. \ref{fig:tri_baudh} for all problems in this section.
\begin{figure}[!ht]
	\begin{center}
		\resizebox{\columnwidth}{!}{%Code by GVV Sharma
%December 7, 2019
%released under GNU GPL
%Proof of Baudhyana Theorem

\begin{tikzpicture}
[scale=2,>=stealth,point/.style={draw,circle,fill = black,inner sep=0.5pt},]

%Triangle sides
\def\a{4}
\def\c{3}
\def\b{sqrt(\a^2+\c^2)}

%Trigonometric ratios
\def\ct{\a/\b}
\def\st{\c/\b}

%perp distance
\def\r{\a*\st}

%Section Ratio
\def\k{1.2}


%Labeling points
\node (A) at (0,\c)[point,label=above right:$A$] {};
\node (B) at (0, 0)[point,label=below left:$B$] {};
\node (C) at (\a, 0)[point,label=below right:$C$] {};

%Foot of perpendicular

\node (D) at ($({\r*\st}, {\r*\ct})$)[point,label=above right:$D$] {};


%Drawing triangle ABC
\draw (A) -- node[left] {$\textrm{c}$} (B) -- node[below] {$\textrm{a}$} (C) -- node[above,xshift=2mm] {$\textrm{b}$} (A);

%Joining BD
\draw (B)--(D);

%Drawing and marking angles
\tkzMarkAngle[fill=orange!40,size=0.5cm,mark=](A,C,B)
\tkzMarkAngle[fill=orange!40,size=0.4cm,mark=](D,B,A)
\tkzMarkAngle[fill=green!40,size=0.5cm,mark=](B,A,C)
\tkzMarkAngle[fill=green!40,size=0.5cm,mark=](C,B,D)
\tkzMarkRightAngle[fill=blue!20,size=.2](A,B,C)
\tkzMarkRightAngle[fill=blue!20,size=.2](B,D,A)
\tkzLabelAngle[pos=0.65](A,C,B){$\theta$}
\tkzLabelAngle[pos=0.65](A,B,D){$\theta$}
\tkzLabelAngle[pos=1](B,A,C){\rotatebox{-45}{$\alpha = 90\degree -\theta$}}
\tkzLabelAngle[pos=0.65](C,B,D){$\alpha$}

\end{tikzpicture}
}
	\end{center}
	\caption{Baudhayana Theorem}
	\label{fig:tri_baudh}	
\end{figure}
\renewcommand{\theequation}{\theenumi}
\begin{enumerate}[label=\thesection.\arabic*.,ref=\thesection.\theenumi]
\numberwithin{equation}{enumi}

%
\item
Show that 
%
\begin{equation}
\label{ch1_budh_basic}
b = a \cos \theta + c \sin \theta
\end{equation}
%
\solution We observe that
%
\begin{align}
BD &= a \cos \theta \\
AD &= c \cos\alpha = c \sin \theta \quad \brak{\text{From} \quad \eqref{eq:tri_90-ang}
}
\end{align}
%
Thus,
\begin{equation}
BD + AD = b = a \cos \theta + c \sin \theta
\end{equation}
\item
From \eqref{ch1_budh_basic}, show that
%
\begin{equation}
%
\label{eq:tri_sin_cos_id}
\sin ^2 \theta + \cos ^2 \theta = 1
\end{equation}
%
\solution Dividing both sides of \eqref{ch1_budh_basic} by $b$, 
\begin{align}
1 &= \frac{a}{b}\cos\theta + \frac{c}{b}\sin\theta\\
\Rightarrow &\sin ^2 \theta + \cos ^2 \theta = 1 \quad \brak{\text{from} \quad \eqref{eq:tri_trig_defs}}
\end{align}
\item In a right angled triangle, the hypotenuse is the longest side.
\label{them:hyp_largest}
\\
\solution From 
\eqref{eq:tri_sin_cos_id},
\begin{align}
	0 \le \sin \theta, \cos \theta \le 1
\end{align}
Hence, 
\begin{align}
	b \sin \theta \le b \implies  c \le b
\end{align}
Similalry,
\begin{align}
	a \le b
\end{align}

\item
	Using \eqref{ch1_budh_basic}, show that
	\begin{equation}
	\label{eq:tri_baudh}
	b^2 = a^2 + c^2
	\end{equation}
	\eqref{eq:tri_baudh} is known as the Baudhayana theorem.  It is also known as the Pythagoras theorem.
\\
\solution From \eqref{ch1_budh_basic},
\begin{align}
b &= a\frac{a}{b} + c \frac{c}{b} \quad \brak{\text{from} \quad \eqref{eq:tri_trig_defs}}\\
\implies b^2 &= a^2 + c^2
\end{align}
\end{enumerate}
%
\iffalse
\section{Applications}
\begin{enumerate}[label=\thesection.\arabic*.,ref=\thesection.\theenumi]
\numberwithin{equation}{enumi}
\item Show that $c > a, c > b$
%
	\\
\solution From 	\eqref{eq:tri_baudh},
	\begin{align}
	c^2 - a^2 &= b^2
\\
\implies c-a &= \frac{b^2}{c+a} > 0 
\\
\implies c &> a
	\end{align}
%
Similarly, it can be shown that $c > b$.
\iffalse
\item Draw Fig. \ref{fig:tri_baudh} for $a = 4, c =3$.
\label{const:tri_baudh}
%
\\
\solution Problem \ref{const:tri_right_angle} is used to draw $\triangle ABC$.
%
Using Problem \ref{prob:tri_polar},
\begin{align}
\vec{D} &= BD\myvec{\cos \alpha\\  \sin \alpha} 
&= a \sin \theta \myvec{ \sin \theta \\ \cos \theta } 
\label{eq:tri_baudh_foot}
\end{align}
%
Using \eqref{eq:tri_baudh_foot}, the python code for  Fig. \ref{fig:tri_baudh} is
\begin{lstlisting}
codes/triangle/tri_baudh.py
\end{lstlisting}
%
and the equivalent latex-tikz code is
%
\begin{lstlisting}
figs/triangle/tri_baudh.tex
\end{lstlisting}
%
\item Using 	\eqref{eq:tri_baudh}, for $a = 4, c = 3$,
%
\begin{align}
b = \sqrt{a^2+c^2} = \sqrt{4^2+3^2} = 5
\end{align}
%
\item For  point $\vec{D} = \myvec{d_1\\d_2}$, its {\em norm} is defined as
%
\begin{align}
OD = d_1^2+d_2^2 = \norm{\vec{D}} \define \sqrt{\vec{D}^{\top}\vec{D}}, 
\label{eq:tri_norm_def}
\end{align}
%
where 
%
\begin{align}
\label{eq:tri_transpose_def}
 \vec{D}^{\top}  \define \myvec{d_1 & d_2},
\\
\vec{D}^{\top}\vec{D} \define \myvec{d_1 & d_2} \myvec{d_1 \\ d_2} = d_1^2+d_2^2
\end{align}
%
\eqref{eq:tri_transpose_def} is the definition of {\em transpose}. $\vec{D}$ is defined to be a {\em column vector} and $\vec{D}^{\top}$  is the corresponding {\em row vector} representing the same point.

\item Also, it is easy to verify that
%
\begin{align}
\label{eq:tri_norm_dist}
AC \define  \norm{\vec{A}-\vec{C}} =  \norm{\myvec{4\\-3}} = \sqrt{3^2+4^2} = 5
\end{align}
%
This is known as the {\em distance formula}.
\fi
%
\item Prove the distance formula in 
  \eqref{eq:norm2d_dist}
 using the Baudhayana theorem.
%
\item Show that 
\label{them:tri_baudh_orth}
\begin{align}
\label{eq:tri_baudh_orth}
\brak{\vec{A}-\vec{B}}^{\top}\brak{\vec{B}-\vec{C}} = 0
\end{align}
\\
\solution From the Baudhayana theorem,
\begin{align}
a^2+c^2 &= b^2
\\
\implies \norm{\vec{B}-\vec{A}}^2+\norm{\vec{C}-\vec{A}}^2&=\norm{\vec{B}-\vec{C}}^2
\label{eq:tri_baudh_orth_norm}
\end{align}
which, from 
  \eqref{eq:norm2d}
 can be expressed as
\begin{multline}
\brak{\vec{B}-\vec{A}}^{\top}\brak{\vec{B}-\vec{A}}
+
\brak{\vec{C}-\vec{B}}^{\top}\brak{\vec{C}-\vec{B}}
\\
=
\brak{\vec{A}-\vec{C}}^{\top}\brak{\vec{A}-\vec{C}}
\end{multline}
%
Expanding
\begin{multline}
\brak{\vec{B}-\vec{A}}^{\top}\brak{\vec{B}-\vec{A}} 
%\\
= \vec{B}^{\top}\vec{B} - \vec{B}^{\top}\vec{A} - \vec{A}^{\top}\vec{B}+\vec{A}^{\top}\vec{A}
\end{multline}
$\because \vec{A}^{\top}\vec{B} = \vec{B}^{\top}\vec{A}$, the above equation can be expressed as
\begin{align}
\norm{\vec{B}-\vec{A}}^2 = 
\norm{\vec{A}}^2 + \norm{\vec{B}}^2 - 2\vec{A}^{\top}\vec{B}
\end{align}
%
Thus, \eqref{eq:tri_baudh_orth_norm} can be expressed using the above equation as
\begin{multline}
\norm{\vec{A}}^2 + \norm{\vec{B}}^2 - 2\vec{A}^{\top}\vec{B}
%\\
+
\norm{\vec{B}}^2 + \norm{\vec{C}}^2 - 2\vec{B}^{\top}\vec{C}
\\
=
\norm{\vec{A}}^2 + \norm{\vec{C}}^2 - 2\vec{A}^{\top}\vec{C}
\end{multline}
%
which can be simplified to obtain
%
\begin{align}
2\norm{\vec{B}}^2 - 2\vec{B}^{\top}\vec{C}
%\\
- 2\vec{A}^{\top}\vec{B}+ 2\vec{A}^{\top}\vec{C}
=0
\\
\text{or, } \vec{B}^{\top}\brak{\vec{B}-\vec{C}}
-\vec{A}^{\top}\brak{\vec{B}-\vec{C}} = 0
\\
\implies \brak{\vec{B}^{\top}-\vec{A}^{\top}}\brak{\vec{B}-\vec{C}} = 0
\end{align}
yielding \eqref{eq:tri_baudh_orth}.
\end{enumerate}
\fi
