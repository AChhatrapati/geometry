%\renewcommand{\theequation}{\theenumi}
%\begin{enumerate}[label=\arabic*.,ref=\theenumi]
\begin{enumerate}[label=\thesection.\arabic*.,ref=\thesection.\theenumi]
\numberwithin{equation}{enumi}
\item Suppose the equations $AB, BC$ and $CA$ are respectively given by 
		\begin{align}
			\label{eq:tri-sides}
			\vec{n}_i^{\top}\vec{x}=c_i \quad i = 1, 2, 3 
		\end{align}
		The equations of the respective angle bisectors are then given by 
		\begin{align}
			\frac{\vec{n}_i^{\top}\vec{x}-c_i}{\norm{\vec{n}_i}}
		=
	\pm	\frac{\vec{n}_j^{\top}\vec{x}-c_j}{\norm{\vec{n}_j}}
\quad i \ne j
		\end{align}
		Substitute numerical values and find the equations of the angle bisectors of $A, B$ and $C$.
	\item Find the intersection $\vec{I}$ of the angle bisectors of $B$ and $C$.
	\item Using 
    \eqref{eq:angle2d}
verify that 
		\begin{align}
			\angle BAI = \angle CAI.
		\end{align}
	\item Find the distance from $\vec{I}$ to $BC$.  
	\item Repeat the above exercise for the sides $AB$ and $AC$.
	\item This distance is known as the {\em inradius} $r$.
	\item Draw a circle with center $\vec{I}$ and radius $r$.  $\vec{I}$ is known as the {\em incentre}.
	\item The equation of the {\em incircle} is given by 
		\begin{align}
			\norm{\vec{x}-\vec{I}}^2 = r^2
		\end{align}
		Find the parameteric equation of $BC$ and use it to verify that $BC$ intersects the incircle at exactly one point $\vec{D}_3$.  $BC$ is defined to be a {\em tangent} to the incircle.  $\vec{D}_3$ is defined to be {\em point of contact}.
	\item Find the other points of contact $\vec{E}_3$ and $\vec{F}_3$.
	\item Verify that 
		\begin{align}
			AE_3 = AF_3=m, BD_3 = BF_3=n, CD_3 = CE_3=p.
		\end{align}
	\item Obtain $m,n,p$ in terms of $a,b,c$, the sides of the triangle using a matrix equation.  Obtain the numerical values.
\end{enumerate}
